\section{Introduction}

% Pashto poetry, with its rich history and cultural significance, plays an essential role in the literary and oral traditions of the Pashtun people. From classical forms such as Ghazals and Rubaiyat to modern free verse, Pashto poetry reflects the social, emotional, and philosophical experiences of its speakers. However, despite its importance, Pashto remains a low-resource language in the context of computational linguistics, with limited availability of annotated data and tools for natural language processing (NLP). This gap poses challenges for preserving and promoting Pashto literature on a global scale.

% The increasing reliance on technology for cultural preservation presents new opportunities to bridge this gap. Deep learning techniques, particularly those focused on text classification, offer a robust solution for analyzing and organizing literary texts in underrepresented languages like Pashto. Automated classification of poetry based on themes and structures not only helps in safeguarding the cultural essence of the language but also provides tools for broader academic and linguistic research. Furthermore, such advancements can facilitate improvements in machine translation, making Pashto more accessible to non-native speakers and researchers.

% In this paper, we introduce a deep learning-based approach to classify Pashto poetry, focusing on its potential to promote linguistic and cultural understanding. By employsing neural networks, our system categorizes poetry into various themes and styles, providing insights into the structure and content of Pashto literary works. Beyond classification, this research offers valuable contributions to the field of digital humanities, where technology intersects with cultural studies. Additionally, the project paves the way for further advancements in NLP and machine translation for Pashto, addressing the challenges faced by low-resource languages in the digital age.

Poetry has been a cornerstone of cultural and literary heritage across the globe, as it is one of the most popular ways to express emotions, thoughts, and experiences. Pashto is one of the major languages spoken in Afghanistan and Pakistan, with 50\% of the Afghan population speaking Pashto, and 15\% of the Pakistani population speaking Pashto \cite{ciawordlfactbook} \cite{pbs}. As such, it also holds a significant place in the rich tapestry of South Asian literature with its first recorded poetic works believed to be dating back to the 8th century with the works of Amir Kror Suri (a warrior poet) \cite{history_pashto}. There have also been other notable poets such as Khushal Khan Khattak, Rahman Baba, Ghani Khan, and Hamza Baba who have contributed to the Pashto literary tradition, and are revered for their works. However, despite its rich history and cultural significance, Pashto remains a low-resource language in the context of computational linguistics, with limited availability of annotated data and tools for natural language processing (NLP) apart from translated works. This gap poses challenges for preserving and promoting Pashto literature on a global scale. Thus, the attribution of poetic works to specific poets can sometimes be ambiguous, due to oral traditions, and lack of proper documentation. 

In recent years, deep learning techniques have gained substantial traction in the field of language processing, text classification, and poet attribution. Poet attribution refers to the process of identifying the poet who authored a specific poem. This process is crucial for preserving the cultural heritage of a language, as well as for academic research and literary studies. While researchers have explored poetry attribution in various languages such as English, Hindi, Urdu, Arabic, Persian, and Gujrati, Pashto poetry has not been extensively studied in this context. This gap could also be attributed to the lack of resources, and documentation in Pashto. This paper aims to bridge this gap by introducing deep learning based approaches to classify Pashto poetry and attribute it to specific poets, which can also prevent misinformation and mis-attribution due to the lack of proper documentation available. This paper also aims to curate the first Pashto poetry dataset of 13 poets (old and relatively newer poets) with about 12000 couplets in total, which will be made publicly available to the research community.