\section{Introduction}

Poetry has been a cornerstone of cultural and literary heritage across the globe, as it is one of the most popular ways to express emotions, thoughts, and experiences. Pashto is one of the major languages spoken in Afghanistan and Pakistan, with 50\% of the Afghan population speaking Pashto, and 15\% of the Pakistani population speaking Pashto \cite{ciawordlfactbook} \cite{pbs}. As such, it also holds a significant place in the rich tapestry of South Asian literature with its first recorded poetic works believed to be dating back to the 8th century with the works of Amir Kror Suri (a warrior poet) \cite{history_pashto}. There have also been other notable poets such as Khushal Khan Khattak, Rahman Baba, Ghani Khan, and Hamza Baba who have contributed to the Pashto literary tradition, and are revered for their works. However, despite its rich history and cultural significance, Pashto remains a low-resource language in the context of computational linguistics, with limited availability of annotated data and tools for natural language processing (NLP) apart from translated works. This gap poses challenges for preserving and promoting Pashto literature on a global scale. Thus, the attribution of poetic works to specific poets can sometimes be ambiguous, due to oral traditions, and lack of proper documentation. 

In recent years, deep learning techniques have gained substantial traction in the field of language processing, text classification, and poet attribution. Poet attribution refers to the process of identifying the poet who authored a specific poem. This process is crucial for preserving the cultural heritage of a language, as well as for academic research and literary studies. While researchers have explored poetry attribution in various languages such as English, Hindi, Urdu, Arabic, Persian, and Gujrati, Pashto poetry has not been extensively studied in this context. This gap could also be attributed to the lack of resources, and documentation in Pashto. This paper aims to bridge this gap by introducing deep learning based approaches to classify Pashto poetry and attribute it to specific poets, which can also prevent misinformation and mis-attribution due to the lack of proper documentation available. This paper also aims to curate the first Pashto poetry dataset of 23 poets (old and relatively newer poets) with 27,607 couplets in total, which will be made publicly available to the research community.