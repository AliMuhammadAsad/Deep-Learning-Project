\section{Discussion and Analysis}
In this section we discuss the results of the experiments and analyze the performance of the models. We also discuss the limitations of the models and the dataset used in the experiments.

We had initially expected the Deep Learning and Transformer Models to perform better on the data. Unexpectedly, however, two of the machine learning models; Logistic Regression and Support Vector Machine, outperformed the Deep Learning and Transformer models. The Logistic Regression model had the highest accuracy of 62.88\% and the Support Vector Machine model had the second highest accuracy of 61.92\%. The Bi-LSTM model trained on the TF-IDF features that were used for the machine learning model performed the third best with an accuracy of 59.63\%. We suspect this might be due to the inherent similar features of the various classes within the data. As such, the TF-IDF feature extractor might be better at capturing these features than the Transformer and Deep Learning models. Also an notable observation we made from our confusion matrices is that all the models tend to follow a similar pattern in terms of accuracy for the different classes. The older poets, mainly, Allama Abdul Hai, Ghani Khan, Kushal Khan Khattak, Mumtaz Orakazi, Rahman Baba, and Rehmat Shah had the highest accuracy across all models. That is, all of the models predicted these poets with the highest accuracy, and all the models predicted the newer poets very poorly. This also strengthens our belief in the inherent similarity between the data as the older poets might use a different style of poetry, while the newer poets might use a modern and similar style of poetry which might be harder to differentiate. Another reason for the poor performance could be the scarcity of data. All models predict Kushal Khan Khattak with the highest accuracy, which might be due to the fact that he has the most number of couplets in the dataset - 4257. All other poets have couplet counts less than 2000, apart from 2 poets with a couplet count of 2380 and 2202. This vast distribution of data might have caused the models to overfit on the data of Kushal Khan Khattak, and not learn the features of the other poets well enough. The data distribution can be referred to from \hyperref[fig:poet-dist-hist]{Section III.A}. 