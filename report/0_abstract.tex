% \begin{abstract}

%     % This paper explores the application of deep learning to the classification of Pashto poetry, aiming to preserve its literary heritage and enhance global appreciation. By leveraging neural networks, the study categorizes Pashto poems based on thematic and structural features, offering insights into its poetic forms. This classification facilitates deeper linguistic analysis and educational engagement, while also advancing machine translation and natural language processing (NLP) for Pashto, a low-resource language. Additionally, the research contributes to digital humanities by enabling large-scale analysis, fostering a bridge between technology and cultural studies.
    
% \end{abstract}

